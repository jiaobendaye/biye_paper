% !Mode:: "TeX:UTF-8"

\hitsetup{
  %******************************
  % 注意:
  %   1. 配置里面不要出现空行
  %   2. 不需要的配置信息可以删除
  %******************************
  %
  %=====
  % 秘级
  %=====
  statesecrets={公开},
  natclassifiedindex={TM301.2},
  intclassifiedindex={62-5},
  %
  %=========
  % 中文信息
  %=========
  % ctitleone={局部多孔质气体静压},%本科生封面使用
  % ctitletwo={轴承关键技术的研究},%本科生封面使用
  ctitlecover={基于群体智能的基因表达数据双聚类的研究},%放在封面中使用,自由断行
  ctitle={基于群体智能的基因表达数据双聚类的研究},%放在原创性声明中使用
  %csubtitle={一条副标题}, %一般情况没有,可以注释掉
  cxueke={工学},
  csubject={计算机软件与理论},
  caffil={计算机与信息科学学院},
  cauthor={凡振豪},
  csupervisor={欧灵副教授},
  % cassosupervisor={某某某教授}, % 副指导老师
  % ccosupervisor={某某某教授}, % 联合指导老师
  % 日期自动使用当前时间,若需指定按如下方式修改:
  %cdate={超新星纪元},
  % cstudentid={9527},
  % cstudenttype={同等学力人员}, %非全日制教育申请学位者
  %(同等学力人员)、(工程硕士)、(工商管理硕士)、
  %(高级管理人员工商管理硕士)、(公共管理硕士)、(中职教师)、(高校教师)等
  %
  %
  %=========
  % 英文信息
  %=========
  etitle={Research on  biclustering of gene expression data based on swarm intelligence},
  esubtitle={This is the sub title},
  exueke={Engineering},
  esubject={Computer Architecture},
  eaffil={\emultiline[t]{College of Computer and Information Science}},
  eauthor={FAN Zhenhao},
  esupervisor={OU Ling},
  % eassosupervisor={XXX},
  % 日期自动生成,若需指定按如下方式修改:
  % edate={December, 2017},
  estudenttype={Master of Engineer},
  %
  % 关键词用“英文逗号”分割
  ckeywords={群智能算法;基因表达数据;双聚类;多目标优化},
  ekeywords={Swarm Intelligence Algorithm,Gene Expression Data, Biclustering, Multiple-optimistic},
}

\begin{cabstract}
  高通量基因微阵列技术的出现, 产生了大量的基因表达数据。这些数据在追踪生物过程,基因规则发现以及病理分析中有着至关重要的作用。基因表达数据的双聚类是指, 找出在某些条件子集下包含一致表达波动的基因子集。双聚类可以看作是一种优化问题。针对表达数据高维度, 高冗余的特点, 许多群智能算法被用于双聚类中来。本文对粒子群算法、量子粒子群算法, 萤火虫算法以及布谷鸟算法在四个数据集上进行了比较研究。实验结果表明了量子粒子群算法与布谷鸟算法具有更好的全局寻优能力。最后基于基因本体工程, 对双聚类结果给出了生物学解释。


\end{cabstract}

\begin{eabstract}
  The advent of high-throughput gene microarray technology has generated a large amount of gene expression data. These data play a vital role in tracking biological processes, discovering genetic rules, and analyzing pathology. Biclustering of gene expression data refers to finding a subset of genes that contain consistent expression fluctuations under certain conditional subsets. Biclustering can be considered as an optimization problem. In view of the high dimensionality and high redundancy of expression data, many swarm intelligence algorithms are used in biclustering. In this paper, particle swarm algorithm, quantum particle swarm algorithm, firefly algorithm and cuckoo algorithm are compared and researched on four data sets. The experimental results show that the quantum particle swarm algorithm and cuckoo algorithm have better global optimization capabilities. Finally, based on the gene ontology engineering, a biological explanation is given for the results of the biclustering.

\end{eabstract}