% !Mode:: "TeX:UTF-8" 
\begin{conclusions}
\section*{一、本文工作总结}
信息科学技术的持续创新与发展使得互联网音乐盛行。各个音乐平台的不断更新和优化,纷纷竞争着迎合市场需求,争取留住老用户,赢得新用户。音乐流行趋势直接决定了音乐未来走势,这都需要音乐平台的开发者发掘新的算法和技术,让用户体验更好,从而增加平台收益。本文切实从互联网企业的实际问题出发,针对互联网音乐平台竞争力的不断增加的问题,综合分析用户的行为记录和歌手歌曲数据,预测歌手们的播放量即进行预测音乐流行趋势,以把握未来一段时间内的流行歌曲、流行歌手,这对于音乐平台来说,预先把握了流行歌曲,那么就会获得用户,增长收益;对于歌手而言,可以被提前挖掘出来,为自己美好的发展作准备;对于用户来说,能听到精准预测得到的流行歌手的流行歌曲,会有很满意的体验。

目前,对于互联网音乐流行趋势预测的研究主要存在以下几种模型和算法:
\begin{itemize}
    \item[(1)] {利用神经网络模型进行音乐流行趋势预测。这种模型的构建速度有待提高,对实验环境要求高,使得耗费时间多,预测效果也一般。而且人工神经网络需要大量的参数,那么参数的设定对于实验的结果影响较大,且工作量繁杂。}
    \item[(2)] {利用支持向量机模型的音乐流行趋势预测。这个模型对于大规模的数据训练比较困难,无法直接支持多分类。模型的学习能力有限,对于核函数的高维映射解释力不强,尤其是径向基函数。对缺失数据敏感。对于处理大量的互联网音乐数据来说,不是一个好的选择。}
    \item[(3)] {利用ANN+SVM组合模型进行预测音乐流行趋势。这种模型比的单一的模型效果要好,但仍然不能很好的学习多维特征,处理大量数据训练能力不佳,因此对音乐流行趋势预测的效果提升不高。}
    \item[(4)] {通过GBDT模型进行音乐流行趋势预测。模型对更换数据集较为敏感,对于任意足够大,且总体参数不稳定的数据集,预测效果就差很多。模型对自身参数也敏感,不能保证参数在一定范围内的变化不敏感,预测效果一般。}
\end{itemize}

本文针对音乐流行趋势预测模型目前的研究成果所面临的挑战,研究了基于随机森林算法预测音乐流行趋势模型和爆增型歌曲衰减模型,基于LSTM时间序列预测音乐流行趋势模型以及基于ARIMA时间序列预测音乐流行趋势模型。本文主要贡献有以下几点:

\begin{itemize}
    \item[(1)] {对互联网音乐数据进行了分析、特征选择以及编码处理,对异常的数据进行了筛查、清洗。分析了音乐数据的特点,从用户、歌手、歌曲这三个角度分别分析和处理数据,然后综合研究音乐流行趋势预测的问题。由于互联网音乐数据特征多维,类别各样,因此会针对性的对部分数据进行独热编码(One-Hot Encoding)处理。}
    \item[(2)] {构建了基于随机森林算法的音乐流行趋势预测模型。结合音乐数据编码后特征类别多样的特点,随机森林算法的表现性能好,能很好处理高维度的数据,训练速度快,抗干扰能力强等优点,构建随机森林算法预测音乐流行趋势的模型,通过与GBDT等其他算法进行对比分析,该模型的预测效果较好。同时,考虑到音乐数据的独特性,歌手们会发布的新歌或者开演唱会等特殊情况,通过对播放量爆增的歌曲数据分析可知,播放量爆增的歌曲会有一个衰减的过程,专门针对这部分特殊数据,构建一个爆增型歌曲衰减模型,综合随机森林算法预测模型,组合模型预测能力表现得比纯用随机森林模型预测更好。}
    \item[(3)] {构建了基于LSTM时间序列预测音乐流行趋势模型和基于ARIMA的时间序列预测音乐流行趋势模型。通过对数据特点的了解,结合LSTM具备长期记忆功能,可控记忆能力强等优势,构建了一种基于LSTM时间序列的音乐流行趋势预测模型。同时,考虑到ARIMA模型简单,只需要内生变量而不需要借助其他外生变量的优势,本文也构建了基于ARIMA的时间序列音乐流行趋势预测模型。对比两个时间序列预测模型,ARIMA模型的预测能力更胜一筹。}
\end{itemize}

实验过程中,模型对更换数据集不敏感,对于任意足够大,且总体参数稳定的数据集,都可以得到较好的预测结果。模型对自身参数也不敏感,可以保证参数在一定范围内的变化,预测能力保持在一个很高的水平上。根据本文提出的评估指标F值,利用音乐数据集对本文的四个音乐流行趋势模型进行测试,纵向对比四个模型的评估指标值。

实验结果表明,基于随机森林算法的音乐流行趋势预测模型相对于两个时间序列模型的评估指标值值要高,预测性能更好;构建的爆增型歌曲衰减模型与随机森林算法的预测模型相结合,预测能力的评估指标值值增加,整体的预测效果更佳,预测能力明显提升。总之,实验结果表明,本文构建的音乐流行趋势预测模型预测音乐流行趋势是有效的。
\section*{二、未来工作展望}
论文综合考虑了音乐数据的特点,随机森林算法的优点和时间序列模型的优势研究了音乐流行趋势预测模型,取得了一定的研究成果,但是仍然存在一些问题和难点需要去攻克。
\begin{itemize}
    \item[(1)]  {音乐数据的周期性规律很弱且数据量大,这导致很难对每个歌手两个月播放量进行一一详细预测。比较容易实现的是将每个歌手的两个月播放量统一预测。但是,仍然存在其他突发性事件(如电视剧歌曲的热播等)会导致部分歌手的播放量难以预测。}
    \item[(2)]  {随机森林算法回归预测音乐流行趋势模型预测能力较好,而ARIMA的时间序列预测模型相对于LSTM时间序列预测模型预测效果更好,如果融合随机森林预测和时间序列预测两种模型可行,这样的融合模型的预测效果是否更好,有待实验验证}
\end{itemize}


\end{conclusions}
