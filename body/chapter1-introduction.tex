% !Mode:: "TeX:UTF-8"
\chapter{绪论}

\section{研究背景及意义}
    随着科技的进步,人类对自然,以及对生命有了更为深刻的认识。从显微镜的发明到DNA双链结构的提出,再到如今的21世纪,人们越来越多地发现,许多重大疾病跟基因相关,如癌症、一些先天性的心脏病和肥胖症。对于基因的研究成为了解决这些难题的关键。无论是2003年的SARS,还是2019年底的新型冠状病毒,基因都是研究这些病毒的突破点,搞懂了其基因的组成和作用,将极大地帮助研究人员研制出对应的抗体。每次病毒的大规模扩散,对所在国家的经济和发展都会带来巨大的损失,因此,基因研究是一项任重而道远的,属于全人类的任务。

    随着生物实验技术的进步,基因测序变得越来越方便和便宜。在美国,已经有公司推出了亲民的基因测序产品,人们只需花费数百美元就可以测自己的基因,从而可以提前知道将来容易得哪些疾病,并做好预防。在细胞的生物过程中,基因通过转录成信使核糖核酸(mRNA),在不同酶和氨基酸的参与下合成各种各样的蛋白质,这一过程称之为基因的表达。尽管同一生命体中的各个细胞的基因序列是相同的,但不同的细胞仅会在特定的条件下表达特定的极少数基因。所以,研究基因的表达调控是基因组表达分析的重要内容,也有很大的意义,比如,它有助于确认病毒感染和致癌基因,并有助于确认在细胞的各个生命周期内活性基因等。通过高通量基因表达测量技术如微阵列技术(Microarray),可以测得不同mRNA的在细胞中的含量。该数据代表着基因的表达水平,因此被成为基因表达数据。

    在生物信息学中,对基因表达数据的挖掘是研究热点。通过对基因表达数据的聚类分析,可以得到感兴趣的差异基因。这些基因在不同的实验条件(如样本,时间)下,存在某种一致的表达模式。通过对这些基因的富集或旁路分析,找到这些基因的功能以及相互的调控关系。

    然而,基因表达数据有两个主要特点。一,一般而言,基因的数量在几千到几万,而样本或条件的数量只有几十到几百。二,正如前面所说,基因的表达是条件相关的。基因有可能会在多个条件下表达,也有可能在所有的条件下都没有表达。这些特点使得常规的聚类方法无法胜任,于是就引入了双聚类(Bicluster)的概念。不同与常规聚类的全局模式(Global Pattern),双聚类专注于寻找局部模式(Local Pattern),它不要求同一类基因只有在所有实验条件下才具有相似表达,而只要求在部分实验条件下具有相似的表达。找到的基因子集和条件子集就构成了一个双聚类。

\section{相关研究进展}
    双聚类(bicluster)这一单词最早由Hartigan于1972年提出。直到2000年,Cheng和Church将双聚类引入到基因表达数据挖掘中,提出了CC算法,并得到了较好的效果。他们提出了MSR(Mean Square Residue,均方残差)用于评价双聚类相似性。MSR越小,则表明行相似性和列相似性越高。该算法先通过不断地删除基因节点和条件节点,找到小于事先给定的MSR阈值$\delta$的双聚类,然后将其作为初始双聚类,在保证MSR不会增大的前提下,不断向其中添加基因节点和条件节点,最终得到一个双聚类结果。如果想要多个结果,算法会把之前找到的双聚类使用随机数覆盖,再重复上述操作,直到获得想要数量的双聚类。随机数的引入,会导致结果不准确,而且无法找到重叠的双聚类。2002年,Yang等对CC算法进行改进,提出了FLOC算法。算法从多个初始双聚类出发,根据最大增益的原则,来执行基因和条件节点的删除或增加。但并没有解决贪心策略带来的陷入局部最优解的问题。

    2002年,Ihmels等人提出了ISA算法(Iterative Signature Algorithm,迭代签名算法)。ISA并没有使用MSR,而是将双聚类视为转录模块,然后通过对其进行打分,迭代地修改基因集和条件集,直到无法再继续修改。同年,Tanny等提出了SAMBA(Statistical Algorithm Method For Bicluster Analysis)算法。该算法使用了图论和统计学的知识,将基因表达数据看作一个二分图,一边为基因,一边为条件。通过寻找最大稠密子图的方式来找到基因表达数据中的最大子矩阵,即双聚类。

    由于双聚类问题是NP难问题,通过贪婪或枚举的方法很难高效地得出结果,人们开始使用元启发式算法,如群智能算法来寻找双聚类。2004年,stanfan等人设计了一个基因表达数据双聚类分析的进化算法框架。Ken-neth Bryan等人于2006年提出了基于模拟退火算法的双聚类算法,并通过实验证明所提方法获得了质量较优的双聚类。2006年Mitra提出了多目标进化双聚类的框架,应用经典的非支配排序遗传算法( Non-dominated Sorting Genetic Algorithm-II,NSGA-II)算法,整合局部搜索策略,并提出新的定量度量方法估计双聚类的质量。2007年,Goiraldez应用最大标准区域(Maximal Standard Area,MSA)作为度量双聚类的标准,并和MSR一起应用到多目标进化算法MOEA中,提出多目标连续变化双聚类算法 (Sequential Multi objective Biclustering,SMOB),有效解决成比例模式的双聚类问题。2013 年Carlos A. Brizula 提出了一种改进的多目标遗传双聚类算法(Enhanced Multi-objective Genetic Biclustering ,EMOGB),与其它多目标双聚类算法不同的是他采用了一种基因和实验条件组来代表一个双聚类的编码方式, 并且在搜索双聚类的过程中减少了局部搜索环节,从而算法的执行效率有了很大的提高。
    

\section{本文的研究内容及组织结构}
    本文首先对基因表达数据和双聚类分析等相关基础知识进行阐述。然后从单目标优化,混合优化和多目标优化等方面,研究了以群智能算法如布谷鸟搜索算法、萤火虫算法和细菌觅食算法为框架的双聚类算法。最后对本文的工作进行了总结和展望。本文各章节内容安排如下:

    第一章从生物背景知识出发,讨论了基因研究的重要性以及双聚类的作用和发展现状。双聚类克服了常规聚类与基因表达数据之间的矛盾,能更好的挖掘出有价值的基因子集和条件子集。群智能算法由于其快速且效果更好,在双聚类分析中得到了很广泛地应用。

    第二章先是更具体地讨论的基因表达数据的数学形式,以及双聚类的定义、类型和结构。接着将双聚类算法分为了基于质量评价指标和基于模型两种。然后,对于双聚类结果,讨论了其质量验证指标和生物验证指标。最后分别介绍了本文所关注的群智能算法。

    第三章为了解决普通双聚类算法的质量评价指标不足和生物意义不明显的缺点,本文将布谷鸟算法与萤火虫算法进行了融合,并提出了CSFAB双聚类算法。通过实验,确定了融合的最佳方案,并从质量评价指标和生物评价指标的角度进行了比较分析。

    第四章首先介绍了多目标优化的进本概念,接着将细菌觅食算法按照多目标优化和双聚类的特点进行了改进,包括适应值函数,以及互不支配时的比较规则,并提出了MOBFOB算法。最后从适应值曲线和生物指标对算法进行了分析。
    
    第五章对全文工作进行了总结和展望。
