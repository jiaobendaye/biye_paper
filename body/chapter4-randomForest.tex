% !Mode:: "TeX:UTF-8"
\chapter{基于随机森林算法预测音乐流行趋势模型}
现今的数字网络时代,用户听音乐和歌手创作、发布音乐的形式都发生了巨变,用户使用各种音乐网站听音乐,听到喜爱的音乐还能进行下载、收藏、评论、点赞等操作,歌手们可以上传自己的新歌。未来,谁先掌握了音乐流行趋势问题的优先权就抢占了音乐市场,毋庸置疑的是,这也是一个潜力巨大的市场。随着音乐用户的需求不断提高,企业管理者和研究开发者都纷纷想抓住了这一市场契机,不断增加音乐平台新功能,研究预测音乐流行趋势就成了现今研究热点之一。音乐流行趋势很大程度上受当下流行歌手的影响,那么问题就延伸到研究歌手是否是当下流行的歌手,需要根据音乐用户的历史操作记录以及最近一段时间内歌手们音乐的播放量来预测未来一段时间的流行歌手,挖掘出下一段时间内的音乐流行趋势,对于音乐爱好者、歌手、企业投资者都是非常有意义的。

近年来对音乐流行趋势预测的研究主要是从人工神经网络和支持向量机的角度去探索,使用人工神经网络算法需要大量的参数,学习时间比较长,学习简单问题的过程也比较繁杂。训练能力和预测能力存在矛盾,容易出现过拟合问题,人工神经网络算法的预测受到各种因素的影响。如果数据的特征数很多,支持向量机算法的表现会很差,支持向量机并不能直接提供定量的可能性预测,往往利用交叉验证法来实现,这样要付出高昂的代价。而随机森林算法是综合能力较强的算法,能很好处理维度高的数据,训练速度也快,当存在一部分的特征遗失的状况时,仍能维持准确度。从第二章也可知,随机森林算法被广泛地应用于各个领域,在处理预测问题有着显著的优势。本章从互联网音乐数据实际问题出发,构建了一个爆增型歌曲衰减模型,结合随机森林算法具有的能良好地处理高维数据的能力、强训练能力,以及能较好维持预测准确度的优势,来实现音乐流行趋势预测。

\section{问题分析}

\section{所用的实验工具}

\section{随机森林预测音乐流行趋势}

\section{爆增型歌曲衰减模型ESAM的构建与分析}

\section{本章小结}