\chapter{总结与展望}
\section{论文的工作总结}
科学技术的进步使得人类有了更多的途径来认知事物,而数据则是人类认知事物的特殊媒介。随着高新科技的迅猛发展,越来越多的数据被产生出来,挖掘出蕴藏在数据中的价值,将带来巨大的经济和社会价值。而基因表达数据作为一种特殊的数据,保存着生命密码,能够帮助人们更加了解自身和自然,对医疗以及生态保护都有长远的意义。因为生物体中的细胞种类繁多和基因之间互相调控等方面的原因,基因表达数据有着较为复杂,体量大和增长速度快的特点。因此对于基因表达数据的研究一直是生物信息学领域的难点和重点。

聚类一直是人们分析数据时常用的手段之一。而传统的聚类方法无法找到基因表达数据中的局部模式,所以双聚类分析成为了最主要的挖掘工具。在有的双聚类分析算法中,基于元启发式的优化算法和多目标优化被应用到寻找双聚类。因此,本文以解决双聚类算法中常见的问题,如质量评价不高和生物意义不明显,为出发点,运用算法融合和多目标优化的手段,做了一些总结和探索性的工作,主要包括:
\begin{itemize}
    \item[1.]{介绍了基因表达数据双聚类分析的研究背景和意义。然后,对基因表达数据双聚类分析算法的基础概念和和数学定义进行了阐述,如基因表达数据的数学模型,双聚类的相关概念,常见的双聚类分析方法的分类以及常用的双聚类验证指标。最后,对于本文所涉及的群智能算法也进行了说明}  

    \item[2.]{本文基于布谷鸟算法中的levy飞行和萤火虫算法中的亮度吸引作用,将两个算法嵌套融合,提出了CSFAB双聚类算法。然后讨论了适应值函数,混合方案和停止条件,并通过实验证明了所提出算法具有更高的全局搜索能力和收敛能力,达到预期目标。}

    \item[3.]{本文从多目标优化的角度,对细菌觅食算法进行了相应的设计,如种群的趋化操作和迁徙操作。基于多目标细菌觅食算法,提出MOBFOB双聚类算法。算法对双聚类的均方残差和容量同时优化,使得成对立关系的指标能同时得到优化,最终得到占优的双聚类。实验证明,MOBFOB算法能有效地找到高质量的双聚类。}
    
\end{itemize}

\section{后续工作展望}
基因表达数据的双聚类分析是一个很大且复杂的研究方向。由于时间和条件有限,本文仅仅是对群智能算法做了一些工作。本工作仍有不少问题需要进一步的研究,包括:
\begin{itemize}
    \item [1.]{由于性能的关系,质量评价指标和生物意义被分别考虑。优化算法也只是根据质量评价指标来寻优,这与基因表达数据双聚类分析的目的是存在一定的偏差的。最近,Nepomuceno 等使用SSB双聚类算法直接以GO注释信息来指引搜索方向,取得了一定的效果,但仍有很多不足。如何更好地直接利用GO信息来需找双聚类,还需要进一步的研究。}

    \item [2.]{二维基因表达数据无法全面的记录所需要的信息。目前所用的基因表达数据都是二维矩阵形式的,而现实中,基因的表达是有时序关系的。如果将时间序列这一维度引入进来,应能够更完全更科学地找到双聚类。}

    \item [3.]{将算法优势互补的方法不止融合一种,还可以通过集成的方法来提高双聚类结果的质量。在机器学习领域,集成能够将多个相对较弱的模型组合成一个效果更好的模型,如随机森林算法,XGBoost算法等。如何将不同的双聚类算法集成为一个效果更好的算法,仍处于研究初始阶段,仍需要大量的研究跟进。}
    
\end{itemize}