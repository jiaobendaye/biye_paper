\chapter{总结与展望}
\section{论文的工作总结}
科学技术的进步使得人类有了更多的途径来认知事物,而数据则是人类认知事物的特殊媒介。随着高新科技的迅猛发展,越来越多的数据被产生出来,挖掘出蕴藏在数据中的价值,将带来巨大的经济和社会价值。而基因表达数据作为一种特殊的数据,保存着生命密码,能够帮助人们更加了解自身和自然,对医疗以及生态保护都有长远的意义。因为生物体中的细胞种类繁多和基因之间互相调控等方面的原因,基因表达数据有着较为复杂,体量大和增长速度快的特点。因此对于基因表达数据的研究一直是生物信息学领域的难点和重点。

聚类一直是人们分析数据时常用的手段之一。而传统的聚类方法无法找到基因表达数据中的局部模式,所以双聚类分析成为了最主要的挖掘工具。在有的双聚类分析算法中,基于元启发式的优化算法和多目标优化被应用到寻找双聚类。因此,本文以解决双聚类算法中常见的问题,如质量评价不高和生物意义不明显,为出发点,运用算法融合和多目标优化的手段,做了一些总结和探索性的工作,主要包括:
\begin{itemize}
    \item[1.]{介绍了基因表达数据双聚类分析的研究背景和意义。然后,对基因表达数据双聚类分析的基础概念和和数学定义进行了阐述,如基因表达数据的数学模型,双聚类的相关概念,常见的双聚类分析方法的分类以及常用的双聚类验证指标。最后,对于本文所涉及的群智能算法也进行了说明}  

    \item[2.]{本文基于布谷鸟算法中的levy飞行和萤火虫算法中的亮度吸引作用,将两个算法嵌套融合,提出了CSFAB双聚类算法。然后讨论了适应值函数,混合方案和停止条件,并通过实验证明了所提出算法具有更高的全局搜索能力和收敛能力,达到预期目标。}

    \item[3.]{本文从多目标优化的角度,对细菌觅食算法进行了相应的设计,如种群的趋化操作和迁徙操作。基于多目标细菌觅食算法,提出MOBFOB双聚类算法。算法对双聚类的均方残差和容量同时优化,使得成对立关系的指标都能得到优化,最终得到占优的双聚类。实验证明,MOBFOB算法能都有效地找到高质量的双聚类。}
    
\end{itemize}
\section{后续工作展望}
基因表达数据的双聚类分析是一个很大且复杂的研究方向。由于时间和条件有限,本文仅仅是对群智能算法做了一些工作。本工作仍有不少问题需要进一步的研究,包括:
\begin{itemize}
    \item [1.]{由于性能的关系,质量评价指标和生物意义被分别考虑。优化算法也只是根据质量评价指标来寻优,这与基因表达数据双聚类分析的目的是存在一定的偏差的。最近,Nepomuceno 等使用SSB双聚类算法直接以GO注释信息来指导搜索,取得了一定的效果,但仍有很多不足。如何更好地直接利用GO信息来需找双聚类,还需要进一步的研究。}

    \item [2.]{二维基因表达数据无法全面的记录所需要的信息。目前所用的基因表达数据都是二维矩阵形式的,而现实中,基因的表达是有时序关系的。如果将时间序列这一维度引入进来,应能够更完全更科学地找到双聚类。}

    \item [3.]{将算法优势互补的方法不止融合一种,还可以通过集成的方法来提高双聚类结果的质量。在机器学习领域,集成能够将多个相对较弱的模型组合成一个效果更好的模型,如随机森林算法,XGBoost算法等。如何将不同的双聚类算法集成为一个效果更好的算法,仍处于研究初始阶段,仍需要大量的研究跟进。}
    
\end{itemize}